\documentclass[12pt]{book}
\usepackage{amsmath,amssymb,graphicx,wasysym}
\usepackage{psfrag,pstricks,verbatim}
\usepackage[squaren]{SIunits}
\usepackage{multirow}
\usepackage{halloweenmath}
\topmargin=-.5in
\oddsidemargin=0in
\evensidemargin=0in
\textwidth=6.5in
\textheight=9.1in
\parindent=0pt
\pagestyle{empty}
\begin{document}

Names\underline{ Weston Slayton} \\

\begin{enumerate}

\item{ Use \LaTeX\ to typeset the contents of these two pages, \textbf{including your names on the line above}. Your output should look identical to the content here, including a
footnote\footnote{like this!}, a figure generated in \textit{Mathematica}, and a table generated in \textit{Mathematica}. When you submit, combine---\textbf{in this order}---your PDF, your \LaTeX\ source, and your \textit{Mathematica} code into a single PDF and upload to Schoology as always.}

\begin{enumerate}

\item{ The figure below shows the first five polynomial approximations of cos \textit{x}. Create this exact figure, six inches wide, formatted exactly as shown. Do not type all five functions one at a time; rather, use \verb+Table[]+\ the way we discussed in class. }

\begin{center}
\includegraphics[width = 6in]{p1image.pdf}
\end{center}

\vspace{2.5in}

\item{ The \verb+TableForm[]+\ shown below is a trig table which shows sine, cosine, and tan-gent values for integer angles from 0\degree\ through 360\degree\ in steps of 10\degree. Like we said in class, we could save the \verb+TableForm[]+\ as a PDF and import it like we’ve done before.

$$
\begin{array}{||c|c|c|c||}
\hline
 \text{angle} & \text{sine} & \text{cosine} & \text{tangent} \\
 \hline
 0 & 0 & 1.000 & 0 \\
 10 & 0.1736 & 0.9848 & 0.1763 \\
 20 & 0.3420 & 0.9397 & 0.3640 \\
 30 & 0.5000 & 0.8660 & 0.5774 \\
 40 & 0.6428 & 0.7660 & 0.8391 \\
 50 & 0.7660 & 0.6428 & 1.192 \\
 60 & 0.8660 & 0.5000 & 1.732 \\
 70 & 0.9397 & 0.3420 & 2.747 \\
 80 & 0.9848 & 0.1736 & 5.671 \\
 90 & 1.000 & 0 & \text{ComplexInfinity} \\
 100 & 0.9848 & -0.1736 & -5.671 \\
 110 & 0.9397 & -0.3420 & -2.747 \\
 120 & 0.8660 & -0.5000 & -1.732 \\
 130 & 0.7660 & -0.6428 & -1.192 \\
 140 & 0.6428 & -0.7660 & -0.8391 \\
 150 & 0.5000 & -0.8660 & -0.5774 \\
 160 & 0.3420 & -0.9397 & -0.3640 \\
 170 & 0.1736 & -0.9848 & -0.1763 \\
 180 & 0 & -1.000 & 0 \\
 190 & -0.1736 & -0.9848 & 0.1763 \\
 200 & -0.3420 & -0.9397 & 0.3640 \\
 210 & -0.5000 & -0.8660 & 0.5774 \\
 220 & -0.6428 & -0.7660 & 0.8391 \\
 230 & -0.7660 & -0.6428 & 1.192 \\
 240 & -0.8660 & -0.5000 & 1.732 \\
 250 & -0.9397 & -0.3420 & 2.747 \\
 260 & -0.9848 & -0.1736 & 5.671 \\
 270 & -1.000 & 0 & \text{ComplexInfinity} \\
 280 & -0.9848 & 0.1736 & -5.671 \\
 290 & -0.9397 & 0.3420 & -2.747 \\
 300 & -0.8660 & 0.5000 & -1.732 \\
 310 & -0.7660 & 0.6428 & -1.192 \\
 320 & -0.6428 & 0.7660 & -0.8391 \\
 330 & -0.5000 & 0.8660 & -0.5774 \\
 340 & -0.3420 & 0.9397 & -0.3640 \\
 350 & -0.1736 & 0.9848 & -0.1763 \\
 360 & 0 & 1.000 & 0 \\
 \hline
\end{array}
$$

\textbf{Instead}, use \textit{Mathematica} to create a \LaTeX\ \verb+array+\, and copy and paste it into your \verb+.tex+\ document. Then, in \LaTeX,\ add the appropriate formatting.

}

\end{enumerate}

\end{enumerate}

\end{document}
